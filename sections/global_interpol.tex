
В этом случае мы не задаем явно производные в узлах, но ставим условия непрерывности для первой и второй производной.
\begin{equation}
    P_{3,i}' (x_i)=P_{3,i+1}' (x_i) = b_i,\  P_{3,i}'' (x_i )=P_{3,i+1}'' (x_i) = c_i, \ i=1\dots n-1
    \label{Eq:global_interpol}
\end{equation}
Объединив \eqref{Eq:spline_3} и \eqref{Eq:global_interpol}. и подставив $h_i$ вместо $x_i - x_{i-1}$, получим систему уравнений:
\begingroup
\Large
\begin{equation}
\begin{array}{lr}
     a_{i-1} = a_i - b_i h_i +\frac{c_i}{2}h_i^2 - \frac{d_i}{6}h_i^3\ \ \ \  & i = 2,\dots, n  \\
     b_{i-1} = b_i - c_i h_i +\frac{d_i}{2}h_i^2 & i = 2,\dots, n  \\
     c_{i-1} = c_i - d_i h_i & i = 2,\dots, n  \\
     a_i = y_i & i = 1,\dots, n \\
     a_1 - b_1 h_1 = \frac{c_1}{2} h_1^2 - \frac{d_1}{6} h_1^3 = y_0 &
\end{array}
\label{Sys:step0}
\end{equation}
\endgroup

Всего в этой системе $4n - 2$ уравнения, однако неизвестных коэффициентов $a_i,b_i,c_i,d_i$ равняется $4n$.

Для получения двух недостающих уравнений обычно задают дополнительные условия на концах отрезка $[a,b]$. Такие условия называются краевыми.

В качестве краевых условий обычно используются:
\begin{itemize}
    \item  $S_3'(a) = s_0,\  S_3'(b) = s_n$, где $s_0,\ s_n$ задаются явно \\
    Фундаментальный сплайн
    \item $S_3''(a) = S_3''(b) = 0$ \\
    Естественный сплайн
    \item $S_3'''(a) = S_3'''(b),\ S_3''(a) = S_3''(b)$ \\
    Периодический сплайн
\end{itemize}

Наиболее используемый - второй вариант с приведением вторых производных к нулю на границах отрезка.

Таким образом, с помощью наложения дополнительных условий, получаем систему из $4n$ линейных уравнений для $4n$ неизвестных.